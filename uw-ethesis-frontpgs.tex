% T I T L E   P A G E
% -------------------
% Last updated May 24, 2011, by Stephen Carr, IST-Client Services
% The title page is counted as page `i' but we need to suppress the
% page number.  We also don't want any headers or footers.
\pagestyle{empty}
\pagenumbering{roman}

% The contents of the title page are specified in the "titlepage"
% environment.
\begin{titlepage}
        \begin{center}
        \vspace*{1.0cm}

        \Huge
        {\bf Trust Region Methods for Training Neural Networks }

        \vspace*{1.0cm}

        \normalsize
        by \\

        \vspace*{1.0cm}

        \Large
        Colleen Kinross \\

        \vspace*{3.0cm}

        \normalsize
        A thesis \\
        presented to the University of Waterloo \\ 
        in fulfillment of the \\
        thesis requirement for the degree of \\
        Master of Science \\
        in \\
        Computer Science \\

        \vspace*{2.0cm}

        Waterloo, Ontario, Canada, 2017 \\

        \vspace*{1.0cm}

        \copyright\ Colleen Kinross 2017 \\
        \end{center}
\end{titlepage}

% The rest of the front pages should contain no headers and be numbered using Roman numerals starting with `ii'
\pagestyle{plain}
\setcounter{page}{2}

\cleardoublepage % Ends the current page and causes all figures and tables that have so far appeared in the input to be printed.
% In a two-sided printing style, it also makes the next page a right-hand (odd-numbered) page, producing a blank page if necessary.
 


% D E C L A R A T I O N   P A G E
% -------------------------------
  % The following is the sample Delaration Page as provided by the GSO
  % December 13th, 2006.  It is designed for an electronic thesis.
  \noindent
I hereby declare that I am the sole author of this thesis. This is a true copy of the thesis, including any required final revisions, as accepted by my examiners.

  \bigskip
  
  \noindent
I understand that my thesis may be made electronically available to the public.

\cleardoublepage
%\newpage

% A B S T R A C T
% ---------------

\begin{center}\textbf{Abstract}\end{center}

Artificial feed-forward neural networks (ff-ANNs) serve as powerful machine learning models for supervised classification problems. They have been used to solve problems stretching from natural language processing to computer vision. ff-ANNs are typically trained using gradient based approaches, which only require the computation of first order derivatives. In this thesis we explore the benefits and drawbacks of training an ff-ANN with a method which requires the computation of second order derivatives of the objective function. We also explore whether stochastic approximations can be used to decrease the computation time of such a method. A numerical investigation was performed into the behaviour of trust region methods, a type of second order numerical optimization method, when used to train ff-ANNs on several datasets. Our study compares a classical trust region approach and evaluates the effect of adapting this method using stochastic variations. The exploration includes three approaches to reducing the computations required to perform the classical method: stochastic subsampling of training examples, stochastic subsampling of parameters and using a gradient based approach in combination with the classical trust region method. We found that stochastic subsampling methods can, in some cases, reduce the CPU time required to reach a reasonable solution when compared to the classical trust region method but this was not consistent across all datasets. We also found that using the classical trust region method in combination with mini-batch gradient descent either successfully matched (within 0.1s) or decreased the CPU time required to reach a reasonable solution for all datasets. This was achieved by only computing the trust region step when training progress using the gradient approach had stalled. 


\cleardoublepage
%\newpage

% A C K N O W L E D G E M E N T S
% -------------------------------

\begin{center}\textbf{Acknowledgements}\end{center}

I would like to thank my supervisors, Professor Yuying Li and Professor Justin Wan, for all of their help and guidance through this process. I learned a lot from both of you and really appreciate all of the hard work you each put into helping me succeed. I would also like to thank all of my labmates for their encouragement and support.
\cleardoublepage
%\newpage

% D E D I C A T I O N
% -------------------

\begin{center}\textbf{Dedication}\end{center}

This is dedicated to my family.
\cleardoublepage
%\newpage

% T A B L E   O F   C O N T E N T S
% ---------------------------------
\renewcommand\contentsname{Table of Contents}
\tableofcontents
\cleardoublepage
\phantomsection
%\newpage

% L I S T   O F   T A B L E S
% ---------------------------
\addcontentsline{toc}{chapter}{List of Tables}
\listoftables
\cleardoublepage
\phantomsection		% allows hyperref to link to the correct page
%\newpage

% L I S T   O F   F I G U R E S
% -----------------------------
\addcontentsline{toc}{chapter}{List of Figures}
\listoffigures
\cleardoublepage
\phantomsection		% allows hyperref to link to the correct page
%\newpage

% L I S T   O F   S Y M B O L S
% -----------------------------
% To include a Nomenclature section
% \addcontentsline{toc}{chapter}{\textbf{Nomenclature}}
% \renewcommand{\nomname}{Nomenclature}
% \printglossary
% \cleardoublepage
% \phantomsection % allows hyperref to link to the correct page
% \newpage

% Change page numbering back to Arabic numerals
\pagenumbering{arabic}

